\documentclass[sigconf,authorversion,nonacm]{acmart}

%% \BibTeX command to typeset BibTeX logo in the docs
\AtBeginDocument{% \providecommand\BibTeX{{% \normalfont B\kern-0.5em{\scshape i\kern-0.25em
b}\kern-0.8em\TeX}}}


\begin{document} \title{Re-Ranking Dense Retrieval} \author{Jake Norton} \affiliation{ \institution{University of
Otago} \city{Dunedin} \state{Otago} \country{New Zealand} } \email{norja159@student.otago.ac.nz}


%%
%% The abstract is a short summary of the work to be presented in the
%% article.
\begin{abstract} This paper explores advanced retrieval techniques aimed at enhancing the efficiency
    and effectiveness of information retrieval systems through use of graph-based and
    latent feature methodologies. We analyze three distinct approaches: Latent Approximate Document
    Retrieval (LADR), which utilizes latent features to optimize document retrieval; adaptive
    re-ranking with a corpus graph (Gar), which dynamically adjusts document rankings within a
    corpus graph to improve relevance; and adhoc retrieval through traversal of a query-document
    graph, which employs direct traversal methods to optimize query-specific document retrieval.
\end{abstract}


%%
%% The code below is generated by the tool at http://dl.acm.org/ccs.cfm.
%% Please copy and paste the code instead of the example below.
\begin{CCSXML} <ccs2012> <concept> <concept_id>10002951.10003317.10003338.10010403</concept_id>
<concept_desc>Information systems~Novelty in information retrieval</concept_desc>
<concept_significance>500</concept_significance> </concept> </ccs2012> \end{CCSXML}

\ccsdesc[500]{Information systems~Novelty in information retrieval}

% %% Keywords. The author(s) should pick words that accurately describe %% the work being presented.
% Separate the keywords with commas.
\keywords{dense retrieval, approximate k nearest neighbour, adaptive re-ranking, neural re-ranking,
clustering hypothesis}

%%
%% This command processes the author and affiliation and title
%% information and builds the first part of the formatted document.
\maketitle
%% Make note of how the papers are related to one another
\section{Introduction} 


\section{Lexically-Accelerated Dense Retrieval Research } 

\subsection{Research
Question}

\subsection{Main Contributions} 


\section{Effective Adhoc Retrieval Through Traversal of a Query-Document Graph}

    \subsection{Research Question} 

    \subsection{Main Contributions} 


\section{Adaptive Re-Ranking with a Corpus Graph}

\subsection{Research Questions} 

\subsection{Main Contributions} 

\section{Relations between the papers}

\section{Research Q1}

\subsection{Experiment} 

\subsection{Hypothesis} 

\section{Research Q2}

\subsection{Experiment} 

\subsection{Hypothesis} 

\section{Conclusion} 

 \section*{}
%% The next two lines define the bibliography style to be used, and
%% the bibliography file.
\bibliographystyle{ACM-Reference-Format} \bibliography{references}

\end{document} \endinput

